%-------------------------------------------------------------------------------
%	SECTION TITLE
%-------------------------------------------------------------------------------
\clearpage
\cvsection{Cover Letter}


%-------------------------------------------------------------------------------
%	CONTENT
%-------------------------------------------------------------------------------
%\begin{cvparagraph}

%---------------------------------------------------------

\vspace{4ex}

%Dear Hiring Team,

\vspace{2ex}

I have been pursuing technology innovation for 20+ years, where I led countless successful initiatives spanning diverse domains (AI, data architecture, E-commerce, robotics, mobile apps, games) and settings (Start-up B2C environments, Global corporate headquarters, Multi‑million EU publicly funded R\&D projects and countless independent initiatives). Focusing \textbf{only} on AI and Data initiatives, highlights include (blue text are hyperlinks):
 

\begin{itemize}

\item I built \urlll{https://github.com/n42r/muze-ai}{\textbf{MuzeAI}}, an open source LLM-powered music recommendation service prototype driven by a user’s Spotify account which received promising reception from pilot users (it was an exploration for a possible start-up venture), \textbf{2023-24}.

\item I invented \urlll{https://n42r.github.io/phd}{\textbf{The Basin Network}}, a data architectural pattern for data sharing and exchange (think next-generation data mesh). It was targeted at enabling machine learning teams share, remix and build on each others data within medium to large enterprises as well as across organizations. I was awarded a PhD for it and published two papers on the subject (it was also built with the intention to spin into to a deep-tech start-up), \textbf{2020-2023}.

\item I led a \urlll{https://github.com/simutool}{\textbf{4 year initiative}} (team of 5) in which we built a data management SaaS which optimized operational efficiency by 30\% for a €3.5 million Euro EC-funded R\&D project for a consortium of 8-companies in aerospace and manufacturing which included Airbus, \textbf{2015-19}.

\item I led a team to build 2 AI players for a \urlll{https://n42r.github.io/tw}{\textbf{iOS mobile board game}} (backgammon) in a B2C start-up like organization, which resulted in increasing sales and average ratings from 3 to 4.5/5 on the App Store, I also built a complex random generator for the dice algorithm which greatly improved game playability and engagement, \textbf{2011-2013}.

\item I co-founded and co-led a team of 10 to to build a AI software to play soccer to compete in the International RoboCup competition. I designed the most technically complex and critical component of the project, the ball passing algorithm, using markov chain modeling (a subject I had no prior experience in), and wrote the position paper which earned us a spot in the 2012 RoboCup finals qualification round, \textbf{2011-12}.

\item I invented two frameworks to help make robots more intelligent: one was a influenced by the idea of \urlll{https://ebooks.iospress.nl/volumearticle/6006}{\textbf{short-term memory in humans}}, and the other solved a complex philosophical problem related to \urlll{https://doi.org/10.1007/978-3-642-16111-7\_14}{\textbf{robot perception and visual illusion}}, and published my findings in top-tier AI conferences: ECAI \& KI. Earned a masters degree with a grade of excellent in Computer Science and Engineering for it, \textbf{2009-2010}.

\item I built a solution that integrated structured background knowledge into case-based reasoning, with use-cases in early breast cancer detection and heart attack diagnosis, and published a paper about it, earning a bachelors degree with a grade of excellent for it, \textbf{2008} 

\item Two projects I played a critical role in during my undergraduate years (building a robot car to navigate mazes and building a game in Java/Java 3D) won top university competition awards \textbf{2005-2007}.

\end{itemize}

In recent years, my interests and focus have moved to building highly innovative organizations. Since I joined PUMA I have initiated two high profile, high value organizations. 

\begin{itemize}

\item I built a hierarchical organization concept made up of two teams (4+4 and 2 leads + 2 PMs), which resulted in a 20\% improvement to the performance of PUMA.com (about 2 seconds), \textbf{AND} resulted in a 1 year roadmap (4 backlogs) of improvements which have become a major focus of the web development teams at PUMA Global eComm in 2025.

\item I ideated, founded, and wrote the organizational charter for the eComm Tech Cats (ETC), a multi-functional organizational unit included a Tech overview board, an innovation hub, a communication channel between world wide PUMA ecomm regions, and a platform for collaboration with PUMA Digital \& Technology departments and PUMA global IT.


\end{itemize}


My vision and prediction for the future of AI and machine learning...

 
My leadership approach for staying ahead in technology is about exploration and experimentation: not iterations over the same concept (as in software engineering), but cycles exploring completely different solution directions until one hits a promising lead.

This approach is based on my view of the "exponential law" in the modern tech industry, which was driven by my personal experiences as well as a study of the history of the tech industry post-2000+.

When it comes to building teams and organizations, I see it as an interplay between two forces: organizational culture and people. And building a high performance organization is always possible with the right balance of open mindedness, courage, experimentation, and trust. And I believe I have a healthy dose of these four.

My experiences at PUMA global Ecomm (from 2024 to present) have given me a good understanding of the company’s value chain along with its rich world-wide network of regions. 

As Director AI and Machine Learning, I believe I have the experience and traits to deliver maximum growth and value to PUMA’s strategy in general and in helping realize the vision of the AI and Machine Learning division. 

\vspace{5ex}

Sincerely,\\
	
Nasr Kasrin

%\end{cvparagraph}
